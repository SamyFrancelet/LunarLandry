\section{Méthodes}

Pour ce projet, nous sommes partis de grandeurs et de données physique comme la gravité, la masse et l'accélération. Cela nous a permis de construire une base solide de physique pour ensuite venir ajouter les différents éléments de programmation derrière. De manière générale, nous pensons qu'il est important de se concentrer sur les faits réels de la vie avant de commencer à coder des interfaces graphiques ou autres.
Voici la physique que nous avons utilisé :  
\begin{center}
$\overrightarrow{F}= M \cdot \overrightarrow{a}$\\
\end{center}
Qui est transformé pour l’accélération en chute libre par :
\begin{center}
$\overrightarrow{F}_{grav} = M \cdot \overrightarrow{g}$\\
\end{center}
Où g = -9.81 ms-2 sur la terre et g = -1.62 ms-2 et M est la masse de l’objet soumis à la gravité, dans notre cas le vaisseau spatial.
Ensuite, nous avons aussi utilisé le coefficient de frottement en rapport avec la densité de l’air :
\begin{center}
$\overrightarrow{F}_{friction} = -k \cdot \overrightarrow{v}$\\
\end{center}
Où k est le coefficient de friction de l'atmosphère,en l'occurrence k = 0 sur la lune vu qu'il n'y a pas d'air. De ce fait, tant que le vaisseau est en l'air, seul la force de gravité a un impact sur le vaisseau lorsque les moteurs sont éteints.\\
