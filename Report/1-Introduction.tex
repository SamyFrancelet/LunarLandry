\section{Contexte}
Après avoir suivis le cours d’informatique n°1 pendant 10 mois environ, nous avons accumulé beaucoup de matière sur la programmation en Java en passant par des sujets très variés. Il est donc maintenant temps de réaliser un projet sur lequel nous allons appliquer et surtout réunir un bon nombre des éléments appris.
Pour réaliser ce projet, nous avons formé un groupe de trois personnes de la filières Systèmes Industriels et dans notre cas, de la classe bilingue. Voici les membres du groupe : Pablo Stoeri, Landry Reynard et Samy Francelet.
Le jeu implémenté est un Lunar Lander, celui-ci nous a été proposé à la suite de l’interruption des cours en présentiel causée par la crise sanitaire du covid19 et correspond à notre niveau de programmation atteint en cette fin d’année scolaire 2019-2020.

\section{Objectif du document}
Ce document a pour but d’accompagner les personnes qui vont faire fonctionner, contrôler, ou encore modifier notre programme, en leur donnant les informations suivantes :
\begin{itemize}
\item	Explications générales
\item	Spécificités propres à notre code
\item	Problèmes et solutions que nous avons connu durant le codage
\item	Améliorations possibles
\end{itemize}
L’objectif est donc de rendre notre code compréhensible et accessible à n’importe qui, pour autant qu’il ait les compétences nécessaires en Java.


\section{Spécifications}

\section{Problématique}
