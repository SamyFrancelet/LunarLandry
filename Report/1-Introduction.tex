\section{Contexte}
Après avoir suivis le cours d’informatique n°1 pendant 10 mois environ, nous avons accumulé beaucoup de matière sur la programmation en Java en passant par des sujets très variés. Il est donc maintenant temps de réaliser un projet sur lequel nous allons appliquer et surtout réunir un bon nombre des éléments appris.

Pour réaliser ce projet, nous avons formé un groupe de trois personnes de la filières Systèmes Industriels et dans notre cas, de la classe bilingue. Voici les membres du groupe : Pablo Stoeri, Landry Reynard et Samy Francelet.

Le jeu implémenté est un Lunar Lander, celui-ci nous a été proposé à la suite de l’interruption des cours en présentiel causée par la crise sanitaire du covid19 et correspond à notre niveau de programmation atteint en cette fin d’année scolaire 2019-2020.

\section{Objectif du document}
Ce document a pour but d’accompagner les personnes qui vont faire fonctionner, contrôler, ou encore modifier notre programme, en leur donnant les informations suivantes :
\begin{itemize}
\item	Explications générales
\item	Spécificités propres à notre code
\item	Problèmes et solutions que nous avons connu durant le codage
\item	Améliorations possibles
\end{itemize}
L’objectif est donc de rendre notre code compréhensible et accessible à n’importe qui, pour autant qu’il ait les compétences nécessaires en Java.


\section{Spécifications - Problématique}
Le but de ce jeu est de faire atterrir un vaisseau spatial sur une plateforme situé sur la lune. Il faut donc comme en réalité éviter les obstacles et atterrir en douceur. Dans le cas contraire le vaisseau ne supportera pas l’impact et sera détruit.
La problématique est donc la suivante : Coder un Lunar Lander avec ces différentes fonctionnalités.
Pour cela nous avons utilisé la librairie GDX2D créée par Pierre-André Mudry\cite{GDX2D}.

Voici le cahier des charges qui nous a été transmis :
\begin{itemize}
\item	Au lancement du jeu, le vaisseau doit se trouver dans le ciel.
\item	Pour déplacer le vaisseau, il y a trois commandes/moteur à disposition : gauche, droite, vertical vers le haut.
\item	Chaque moteur doit donner une impulsion au vaisseau qui accélère dans la direction voulue, tout en respectant les lois physiques de la gravité et de la densité de l’atmosphère qui ne sont évidemment pas les même que sur terre.
\item	La quantité de carburant disponible dans le vaisseau est limitée. Le carburant est consommé de manière proportionnelle avec le nombre de moteurs activés. C’est-à-dire que chaque moteur a sa propre consommation de carburant et que si l’on active deux moteurs en même temps ils vont chacun consommer du carburant. Une fois que la réserve de carburant est vide, plus aucune action ne va pouvoir être effectuée sur le vaisseau et celui-ci va poursuivre sa trajectoire.
\item	Le vaisseau spatial peut se poser seulement à un endroit précis, prévu à cet effet et il doit se poser en douceur pour éviter l’explosion. S’il touche le relief à un endroit où il n’y est pas autorisé, il va aussi exploser.

Dans le cahier des charges de bases, il n’y a pas de menu de jeu afin de ne pas perdre de temps sur un élément qui n’est pas très intéressant au niveau de la programmation.
\end{itemize}

\section{Méthodologie}
Pour ce projet, nous allons suivre globalement l’ordre qui nous a été donné sur le site d’informatique, mais étant donné que nous sommes trois dans un groupe et qu’il est important de ne pas faire des tâches à double, cet ordre est un peu perturbé.

De même, pour ce travail de groupe, nous avons choisis d’utiliser GitHub pour que tous les trois membres du groupe puissent coder en même temps sur des fonctionnalités différentes et mettre ensuite tout le travail en commun. L’avantage de Git et GitHub est de pouvoir créer une nouvelle fonctionnalité dans une « branche », sans perturber le travail principal de ses collègues.

Il est important pour ce genre de projet de bien s’attribuer des tâches, surtout une fois que la base du programme est faite. Et c’est encore plus important quand il y a des différences de niveau au sein du groupe pour que le travail d’équipe se fasse proprement.
